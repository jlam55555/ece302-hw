\documentclass{article}

\usepackage{amsmath}
\usepackage{mathtools}

\title{ECE302 -- Project 1 Analytical Results}
\author{Steven Lee \& Jonathan Lam}
\date{\today}

\begin{document}
	\maketitle
	
	\noindent Let $X_i$ be the random variable denoting the roll of a single die; let $Y_j=X_1+X_2+X_3$ be the random variable denoting the roll of three die to generate an ability score; let $Z=\max(Y_1,Y_2,Y_3)$ be the random variable denoting the maximum of three trials to generate an ability score (using the ``fun'' method). All $X_i$ are independent uniform IID; thus all $Y_j$ are independent uniform IID, and all $Z_k$ are independent uniform IID.
	
	\begin{enumerate}
		\item 
		\begin{enumerate}
			\item Three Bernoulli trials
			\begin{equation*}
				P(Y=18)=P(X_1=6,X_2=6,X_3=6)=\frac{1}{6}\frac{1}{6}\frac{1}{6}=\frac{1}{6^3}
			\end{equation*}
			
			\item Complement of a binomial distribution
			\begin{equation*}
				P(Z=18)=1-P((Y_1\ne 18)\land (Y_2\ne 18)\land (Y_3\ne 18))=1-\left(1-\frac{1}{6^3}\right)^3
			\end{equation*}
			
			\item Each trait is a Bernoulli trial
			\begin{equation*}
				P(Z_i=18,\ 1\le i\le 6)=(P(Z=18))^6
			\end{equation*}
			
			\item Get the set where all sums are $\le 9$, but at least one is equal to $9$
			\begin{align*}
				P(Z=9)&=P(\text{all sums }\le 9)-P((\text{all sums }\le9)\land(\text{no sums }=9))\\
				&=P((Y_1\le 9)\land (Y_2\le 9)\land(Y_3\le 9))\\
				&-P((Y_1\le 8)\land (Y_2\le 8)\land(Y_3\le 8))\\
				&=\left(\frac{81}{216}\right)^3-\left(\frac{56}{216}\right)^3
			\end{align*}
			\begin{gather*}
				P(Z_i=9,\ 1\le i\le 6)=(P(Z=9))^6
			\end{gather*}
		\end{enumerate}
	
		\clearpage
		\item
		Let $X_i$ denote the random variable representing the hitpoints (hp) of a goblin, and $Y_j$ denote the random variable representing the damage (dmg) of a fireball shot. Note that $X_i\cup Y_j$ are mutually independent.
		
		\begin{enumerate}
			\item Normal expected value
			\begin{align*}
				E[X]&=\sum_{x} xP(X=x)\\
				&=1(0.25)+2(0.25)+3(0.25)+4(0.25)\\
				&=2.5\\
				E[Y]&=\sum_{y}yP(Y=y)\\
				&=2(0.25)+3(0.5)+4(0.25)\\
				&=3\\
				E[Y>3]&=P(Y=4)=0.25
			\end{align*}
			
			\item We can enumerate the pmf by inspection
			\begin{gather*}
				P(X=1)=P(X=2)=P(X=3)=P(X=4)=0.25\\
				P(Y=2)=P(Y=4)=0.25\\
				P(Y=3)=0.5
			\end{gather*}
			
			\item 
			We break up this question using a partition of $Y$.
			\begin{align*}
				P(\text{slay all 6})&=P(Y\ge X_i\ \forall X_i)\\
				&=P((Y\ge X\ \forall X_i|Y=2)\lor (Y\ge X\ \forall X_i|Y=3)\\
				&\ \ \ \ \ \ \ \ \lor (Y\ge X\ \forall X_i|Y=4))\\
				&=\left(\frac{1}{2}\right)^6(0.25)+\left(\frac{3}{4}\right)^6(0.5)+1^6(0.25)
			\end{align*}
			
			\clearpage
			\item
			We can break down the event in question into a partition of three events (note that it is not possible that the surviving troll has $\le 2$ hp or that the firebolt did $4$ dmg):
			\begin{enumerate}
				\item hp of surviving troll $=4$, dmg $=3$, all other trolls have hp $\le 3$
				\item hp of surviving troll $=4$, dmg $=2$, all other trolls have hp $\le 2$
				\item hp of surviving troll $=3$, dmg $=2$, all other trolls have hp $\le 2$
			\end{enumerate}
			The probabilities of these events are easy to calculate.
			
			Using Bayes' rule, we can calculate the posterior pmf of $X$ given that five trolls didn't survive. Let $W$ denote the event that the other five trolls died, and $W=(\text{i})\cup(\text{ii})\cup(\text{iii})\Rightarrow P(W)=P((\text{i}))+P((\text{ii}))+P((\text{iii}))$ (union becomes addition since the events are disjoint). Then:
			\begin{equation*}
				P(X=x|W)=\frac{P((X=x)\land W)}{P(W)}
			\end{equation*}
			This in turn can be used to calculate the expected hp of the surviving troll:
			\begin{gather*}
				P((\text{iii}))=\left(\frac{1}{4}\right)\left(\frac{1}{4}\right)\left(\frac{1}{2}\right)^5\\
				P((\text{ii}))=\left(\frac{1}{4}\right)\left(\frac{1}{2}\right)\left(\frac{3}{4}\right)^5\\
				P((\text{i}))=\left(\frac{1}{4}\right)\left(\frac{1}{4}\right)\left(\frac{1}{2}\right)^5\\
				P(X=3|W)=\frac{P((\text{iii}))}{P(W)}\\
				P(X=4|W)=\frac{P((\text{i})\cup(\text{ii}))}{P(W)}\\
				E[X|W]=\frac{1}{P(W)}\left[(3)P(X=3|W)+(4)P(X=4|W)\right]
			\end{gather*}
			
			\item
			Let $Z_i$ denote the random variable denoting a roll of the 20-sided die (to decide whether Shedjam can hit Keene or not), $W_j$ denote a roll of the 6-sided die (the Sword of Tuition's damage), and $V_k$ denote a roll of the 4-sided die (the Hammer of Tenure Denial's damage).
			
			\begin{align*}
				E[\text{dmg}]&=E[\text{dmg}_{SoT}+\text{dmg}_{HTD}]\\
				&=P(\text{hit}_{SoT})E[\text{dmg}_{SoT}|\text{hit}_{SoT}]+
				P(\text{hit}_{HTD})E[\text{dmg}_{HTD}|\text{hit}_{HTD}]\\
				&=\left(\frac{10}{20}\right)(3.5+3.5)+\left(\frac{10}{20}\frac{10}{20}\right)(2.5)
			\end{align*}
		\end{enumerate}
	\end{enumerate}
\end{document}